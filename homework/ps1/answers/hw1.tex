\documentclass{article}

\usepackage{mathtools}

\DeclarePairedDelimiter\abs{\lvert}{\rvert}%
\DeclarePairedDelimiter\norm{\lVert}{\rVert}%
\makeatletter
\let\oldabs\abs
\def\abs{\@ifstar{\oldabs}{\oldabs*}}

\usepackage[utf8]{inputenc}
\usepackage{pgfplots}

\pgfplotsset{
    discard if not/.style 2 args={
        x filter/.code={
            \edef\tempa{\thisrow{#1}}
            \edef\tempb{#2}
            \ifx\tempa\tempb
            \else
                \def\pgfmathresult{inf}
            \fi
        }
    }
}

\usepackage{subcaption}
\usepackage{gensymb}
\usepackage{amsmath,amsfonts,amssymb,amsthm,epsfig,epstopdf,titling,url,array}
\usepackage{enumerate}
\usepackage[a4paper, total={6in, 8in}]{geometry}

\usepackage{algorithm}
\usepackage[noend]{algpseudocode}

\makeatletter
\def\BState{\State\hskip-\ALG@thistlm}
\makeatother

\newcommand{\mathbbm}[1]{\text{\usefont{U}{bbm}{m}{n}#1}}
\newtheorem{thm}{Theorem}[section]
\newtheorem*{thmt*}{Theorem}
\newtheorem{lem}[thm]{Lemma}
\newtheorem{assumption}{Assumption}
\newtheorem{prop}[thm]{Proposition}
\pgfplotsset{compat=1.15}

\newtheorem{defn}{Definition}[section]
\title{Homework 1 - Xavier Gitiaux}
\author{}
\date{February 2019}

\begin{document}
\maketitle

\section{Exercise 1}
Done!

\section{Exercise 2}

\subsection{}
Denote $c=c_{1}c_{2}c_{3}c_{4}$ the cyphertext. The adversary compute $k= a - c_{1} \% 26$. If $c_{2} - k = b\%26$, it concludes that the password was "abcd"; otherwise, he concludes that the password was "kdmf". 

\subsection{}
Denote $c=c_{1}c_{2}c_{3}c_{4}$ the cyphertext. The adversary compute $k= a - c_{1} \% 26$. If $c_{3} - k = c\%26$, it concludes that the password was "abcd"; otherwise, he concludes that the password was "kdmf". 

\subsection{}
Denote $c=c_{1}c_{2}c_{3}c_{4}$ the cyphertext. The adversary compute $k= a - c_{1} \% 26$. If $c_{4} - k = d\%26$, it concludes that the password was "abcd"; otherwise, he concludes that the password was "kdmf". 

\subsection{}
If the key length is equal to the message length, perfect security is achieved: an adversary cannot distinguish the two passwords (i.e., cannot do better than flipping a coin to distinguish "abcd" and "kdmf"). 

\section{Exercise 3}

\subsection{}
Plaintexts and cyphertexts are of size $n$ bits. 

\subsection{}
Given a pair $(m, c)$, the brute force attack consists in searching the $2^{l}\times 2^{l}$ key space for the pair of keys $(k_{1}, k_{2})$ such that $Enc_{k_{1}, k_{2}}(m) = c$. For each $k_{1}$, the intermediate encrypted value $En_{k_{1}}(m)$ is stored in the $n-$bit available memory $s$, which is then accessed to encrypt with key $k_{2}$. In the worst-case, the total number of loops is $2^{2l}$ and each loop requires $2$ encryptions, so a $O\left( n2^{2l}\right)$ running time if each of the $2 \times 2^{2l}$ encryptions takes $O\left(n\right)$ time. 

\begin{algorithm}[h!]
\caption{Exercise 3: Brute-Force Attack - Question 2}
\begin{algorithmic}[1]
\State \textbf{Input:}  $m$, $n$, a $n-$bit storage $s$, $\{E_{k}(.)\}_{k\in \{0,1\}^{l}}$, $\{D_{k}(.)\}_{k\in \{0,1\}^{l}}$
 

 \For  {i=1..l}
 \For {j=1..l}
 \State$s \gets E_{k[i]}(m)$
 \State $s \gets E_{k[j]}(s)$
 \If {$s==c$}
 \State{\bfseries return} $k[i], k[j]$.
 \EndIf
\EndFor
\EndFor
  

\end{algorithmic}
\end{algorithm}

\subsection{}
Given a pair $(m, c)$, an attack can use the $n 2^{l}$ memory space in the following way:
\begin{itemize}
\item For each key $k_{1}$ in the $2^{l}$ key space, encrypt the plaintext $m$ and store $Enc_{k_{1}}(m)$ along with the key $k_{1}$ in memory (assuming that storing a key is memory free) for example in a hash table.  
\item For each key $k_{2}$ in the $2^{l}$ key space, decrypt the cyphertext $c$ using $Dec_{k_{2}}(c)$ and look for a match in the memory space.
\item If a match is found, return the corresponding $k_{1}$ and $k_{2}$. 
\end{itemize}

The first loop takes $2^{l}$ encryptions, each at a cost $O(n)$. The second loop takes at most $2^{l}$ decryptions, each at a cost $O(n)$. Therefore the total running time is $O(n2^{l})$. 

\begin{algorithm}[h!]
\caption{Exercise 3: Brute-Force Attack - Question 3}
\begin{algorithmic}[1]
\State $ATTACK2DES(m,c)$
\State \textbf{Input:}  $m$, $n$, a $n2^{l}$bit storage $s$, $\{E_{k}(.)\}_{k\in \{0,1\}^{l}}$, $\{D_{k}(.)\}_{k\in \{0,1\}^{l}}$
 
 \State $storage \gets \{\}$

 \For  {i=1..l}
 \State $s[E_{k[i]}(m)] = k[i]$
 \EndFor
 \For {j=1..l}
 \State$c1 \gets D_{k[j]}(c)$
 \If {$storage[c1]$}
 	 \State{\bfseries return} $storage[c1], k[j]$
 \EndIf
\EndFor
   
.
\end{algorithmic}
\end{algorithm}

 

\subsection{}
Given a pair $(m, c)$, an attack can use the $n 2^{l}$ memory space in the following way: search through all the keys $k_{1}$ and for each $k_{1}$, compute $c_{1}=En_{k_{1}}(m)$ and apply the procedure $ATTACK2DES$ of the previous question to $c1$ and $c$. There are at most $2^{l}$ for loops and each iteration requires one encryption and one call to $ATTACK2DES$, so $O(2^{l})$ encryptions and $O(2^{l})$ decryptions. 
Therefore the total running time is $O(n2^{2l})$.  


\begin{algorithm}[h!]
\caption{Exercise 3: Brute-Force Attack - Question 4}
\begin{algorithmic}[1]
\State $ATTACK3DES(m, c)$
\State \textbf{Input:}  $m$, $n$, a $n2^{l}$bit storage $s$, $\{E_{k}(.)\}_{k\in \{0,1\}^{l}}$, $\{D_{k}(.)\}_{k\in \{0,1\}^{l}}$
 

 \For  {i=1..l}
 \State $c1\gets E_{k[i]}(m)$.
 \State $k\gets ATTACK2DES(c1, c)$
 \If {$k$ is not None}
 	 \State{\bfseries return} $k_[i], k[1], k[2]$
 \EndIf
\EndFor
   
.
\end{algorithmic}
\end{algorithm}

\section{Exercise 4}
\subsection{}
The adversary first sends a message $m$ with bit $0$ only and receives a cyphertext $c$. Then he sends two messages $m_{0}=m$ and $m_{1}=11$. The challenger returns $c_{b}$. If $c_{b}=c$, the adversary returns $b=0$; otherwise he returns $b=1$. The adversary distinguishes $b=0$ from $b=1$ with probability $1$, since $m$ and $m_{0}$ will be encrypted to the same cyphertext $m \oplus F_{k}(m)$, while $m_{1}$ encrypts to $m_{1}+F_{k}(m_{1})\neq m \oplus F_{k}(m)$.

\subsection{}
The adversary first sends one message $m$ with bit zero only. He receives a cyphertext $c=r||(H(r) \oplus m \oplus k)$. He takes the last $n$ bits of $c$ and compute $h = (H(r) \oplus m \oplus k) \oplus m = H(r)\oplus k$ and then, deduce the key $k$ as $k=h \oplus H(r)$. Then, the adversary sends two messages $m_{0}\neq m_{1}$. When receiving $c_{b}=r^{'} ||c_{2b}$, if $c_{2b}\oplus H(r^{'})\oplus k = m_{o}$, he returns $b=0$; otherwise, he returns $b=1$. The adversary distinguishes $b=0$ from $b=1$ with probability $1$, since he has obtained the key $k$ in the first phase. 

\section{Exercise 5}

\subsection{}
This is a valid block cypher: for each $k\{0,1\}^{2}$ (i.e. each row), there is no repeat along the row and there are exactly $2^{3}$ columns, so the row represents a permutation from $\{0,1\}^{3}$ to $\{0,1\}^{3}$. 

\subsection{}
This is a valid block cypher since for each $k$ the identity function $E_{k}(x)=x$ is a permutation.

\subsection{}
For each $k\in \{0,1\}^{n}$, $E_{k}^{''}$ is a one-to-one function: first, for $x\neq y$, $E_{k}^{''}(x) = E_{k}^{''}(y)$ implies that $\overline{x} = \overline{y}$ and thus that $x=y$. Moreover, for $y\in \{0,1\}$, $y= E_{k}^{''}(k\oplus \overline{y})$, so $E_{k}^{''}$ is onto. Therefore, for $k\in \{0,1\}^{n}$ $E_{k}^{''}$ is a permutation and $E^{''}$ is valid block cypher. 

\subsection{}
If a distinguisher queries $x_{1}$ and $x_{2}\neq x_{1}$ from an oracle $E^{'}_{k}$ for some secret key $k$ , he gets $x_{1}$ and $x_{2}$ respectively which would happen with negligible probability if $E_{k}$ was randomly drawn from the set of all possible permutations of $\{0, 1\}^{n}$. Therefore, $E_{k}^{'}$ is not a secure block cypher.

\bigskip
Similarly, by querying $x_{1}$ and $x_{2}\neq x_{1}$ from the challenger, , a distinguisher would obtain $y_{1}$ and $y_{2}$ and compute $y = y_{1}\oplus y_{2}$. If $y= x_{1}\oplus x_{2}$, the distinguisher returns $1$ (i.e. the challenger uses $E_{k}^{''}$ for some secret key $k$); otherwise, the distinguisher returns $0$ (the challenger uses a permutation randomly drawn from all permutations of $\{0, 1\}^{n}$). Since $y= x_{1}\oplus x_{2}$ happens with negligible probability with a random permutation, the distinguisher will win with non-negligible probability. 

\end{document}