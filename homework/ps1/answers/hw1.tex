\documentclass{article}

\usepackage{mathtools}

\DeclarePairedDelimiter\abs{\lvert}{\rvert}%
\DeclarePairedDelimiter\norm{\lVert}{\rVert}%
\makeatletter
\let\oldabs\abs
\def\abs{\@ifstar{\oldabs}{\oldabs*}}

\usepackage[utf8]{inputenc}
\usepackage{pgfplots}

\pgfplotsset{
    discard if not/.style 2 args={
        x filter/.code={
            \edef\tempa{\thisrow{#1}}
            \edef\tempb{#2}
            \ifx\tempa\tempb
            \else
                \def\pgfmathresult{inf}
            \fi
        }
    }
}

\usepackage{subcaption}
\usepackage{gensymb}
\usepackage{amsmath,amsfonts,amssymb,amsthm,epsfig,epstopdf,titling,url,array}
\usepackage{enumerate}
\usepackage[a4paper, total={6in, 8in}]{geometry}

\usepackage{algorithm}
\usepackage[noend]{algpseudocode}

\makeatletter
\def\BState{\State\hskip-\ALG@thistlm}
\makeatother

\newcommand{\mathbbm}[1]{\text{\usefont{U}{bbm}{m}{n}#1}}
\newtheorem{thm}{Theorem}[section]
\newtheorem*{thmt*}{Theorem}
\newtheorem{lem}[thm]{Lemma}
\newtheorem{assumption}{Assumption}
\newtheorem{prop}[thm]{Proposition}
\pgfplotsset{compat=1.15}

\newtheorem{defn}{Definition}[section]
\title{Homework 1}
\author{}
\date{January 2019}

\begin{document}
\maketitle


\section{Exercise 3}

\subsection{}
Plaintexts and cyphertexts are of size $n$. 

\subsection{}
Given a pair $(m, c)$, the brute force attack consists in searching the $2^{l}\times 2^{l}$ key space for the pair of keys $(k_{1}, k_{2})$ such that $Enc_{k_{1}, k_{2}}(m) = c$.

\subsection{}
Given a pair $(m, c)$, an attack can use the $n 2^{l}$ memory space in the following way:
\begin{itemize}
\item For each key $k_{1}$ in the $2^{l}$ key space, encrypt the plaintext $m$ and store $Enc_{k_{1}}(m)$ along with the key $k_{1}$ in memory.  
\item For each key $k_{2}$ in the $2^{l}$ key space, decrypt the cyphertext $c$ using $Dec_{k_{2}}(c)$ and look for a match in the memory space.
\item If a match is found, return the corresponding $k_{1}$ and $k_{2}$. 
\end{itemize} 

\subsection{}
Given a pair $(m, c)$, an attack can use the $n 2^{l}$ memory space in the following way:
\begin{itemize}
\item For half of the keys $k_{1}$ in the $2^{l}$ key space, encrypt the plaintext $m$ and store $Enc_{k_{1}}(m)$ along with the key $k_{1}$ in memory.  
\item For half of the keys $k_{3}$ in the $2^{l}$ key space, decrypt the cyphertext $c$ and store $Dec_{k_{3}}(c)$ along with the key $k_{3}$ in memory. 

\item For each key $k_{2}$ in the $2^{l}$ key space, decrypt the cyphertext $c$ using $Dec_{k_{2}}(c)$ and look for a match in the memory space.
\item If a match is found, return the corresponding $k_{1}$ and $k_{2}$. 
\end{itemize}

\end{document}